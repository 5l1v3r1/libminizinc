\documentclass[a4paper,13pt,onecolumn]{article}%{memoir}

\usepackage{listings}
\usepackage{hyperref}
\usepackage{amssymb,latexsym,amsmath,stmaryrd}
\usepackage{xspace}
\usepackage{adjustbox}

\ifx\pdftexversion\undefined
 % \usepackage[dvips]{graphicx}
  \usepackage[dvips,usenames]{color}
\else
 % \usepackage[pdftex]{graphicx}
  \usepackage[pdftex,usenames,dvipsnames]{color}
\fi

% MiniSearch syntax for lstlistings

\definecolor{lightgray}{rgb}{0.97, 0.97, 0.97}
%\definecolor{lightgray}{rgb}{0.83, 0.83, 0.83}
%\definecolor{orange}{HTML}{FF7F00}

% Syntax highlighting for stochastic Minizinc in listings
\lstdefinelanguage{minizinc}
{
morekeywords={
% minizinc keywords
ann, annotation, any, array, assert, bool, constraint, else, endif, enum, float, forall, function,
if, in, include, int, list, of, op, output, par, predicate, record, set,
solve, string, test, then, tuple, type, var, where,
%minimize, maximize % removed for minisearch
% minizinc functions
abort, abs, acosh, array_intersect, array_union,
array1d, array2d, array3d, array4d, array5d, array6d, asin, assert, atan, bool2int, card,
ceil, combinator, concat, cos, cosh, dom, dom_array, dom_size, dominance,
fix, exp, floor, index_set, index_set_1of2,
index_set_2of2, index_set_1of3, index_set_2of3, index_set_3of3, int2float, is_fixed,
join, lb, lb_array, length, let, ln, log, log2, log10, min, max, pow, product, round, set2array,
show, show_int, show_float, sin, sinh, sqrt, sum, tan, tanh, trace, ub, and ub_array,
% minisearch keywords
minisearch, search, repeat, next, commit, print, post, sol, scope, time_limit, break, fail, skip
},
sensitive=false, % are the keywords case sensitive
%morecomment=[l][\bfseries\color{OrangeRed}]{::},
morecomment=[l][\em\color{ForestGreen}]{\%},
%morecomment=[s]{/*}{*/},
morestring=[b]",
}

% settings for listings
\lstset{ %
  backgroundcolor=\color{lightgray},   % choose the background color; you must add \usepackage{color} or \usepackage{xcolor}
  basicstyle=\scriptsize\ttfamily,        % the size of the fonts that are used for the code
  belowskip=-2em,
  breakatwhitespace=false,         % sets if automatic breaks should only happen at whitespace
  breaklines=true,                 % sets automatic line breaking
  captionpos=b,                    % sets the caption-position to bottom
  commentstyle=\color{ForestGreen},    % comment style
%  deletekeywords={...},            % if you want to delete keywords from the given language
  escapeinside={\%*}{*)},          % if you want to add LaTeX within your code
  extendedchars=true,              % lets you use non-ASCII characters; for 8-bits encodings only, does not work with UTF-8
  frame=single,                    % adds a frame around the code
  keepspaces=true,                 % keeps spaces in text, useful for keeping indentation of code (possibly needs columns=flexible)
  keywordstyle=\bfseries\color{blue},       % keyword style
  language=minizinc,                % the language of the code
%  morekeywords={*,...},            % if you want to add more keywords to the set
  numbers=left,                    % where to put the line-numbers; possible values are (none, left, right)
  numbersep=5pt,                   % how far the line-numbers are from the code
  numberstyle=\tiny\color{Gray}, % the style that is used for the line-numbers
  rulecolor=\color{black},         % if not set, the frame-color may be changed on line-breaks within not-black text (e.g. comments (green here))
  showspaces=false,                % show spaces everywhere adding particular underscores; it overrides 'showstringspaces'
  showstringspaces=false,          % underline spaces within strings only
  showtabs=false,                  % show tabs within strings adding particular underscores
  stepnumber=1,                    % the step between two line-numbers. If it's 1, each line will be numbered
  stringstyle=\color{Red},     % string literal style
  tabsize=2,                       % sets default tabsize to 2 spaces
  title=\lstname                   % show the filename of files included with \lstinputlisting; also try caption instead of title
}

%\def\mzninline{\lstinline[basicstyle=\ttfamily,annotationstyle=\normalfont]}
\def\mzninline{\verb}


% setting font family (uncomment if you want default font)
%\renewcommand*\rmdefault{ppl}
%\renewcommand*{\familydefault}{\sfdefault}

\newcommand{\Zinc}{\mbox{\sc Zinc}\xspace}
\newcommand{\MiniZinc}{\mbox{\sc MiniZinc}\xspace}
\newcommand{\MiniSearch}{\mbox{\sc MiniSearch}\xspace}
\newcommand{\FlatZinc}{\mbox{\sc FlatZinc}\xspace}

\newcommand{\mznfznlite}{\texttt{mzn-fzn-lite}}
\newcommand{\mzngecodelite}{\texttt{mzn-gecode-lite}}

\newcommand{\pparagraph}[1]{\vspace{2mm}\noindent\emph{#1.}}

\begin{document}
\title{\MiniSearch Manual}
\author{Andrea Rendl \and Guido Tack}
%National ICT Australia (NICTA) and Faculty of IT, Monash University,
%Australia

\maketitle

\tableofcontents

\section{Overview}
This is the documentation and manual for using \MiniSearch. We give
a brief introduction of \MiniSearch in Sec.~\ref{sec:intro}.
If you are already familiar with using \MiniZinc and want to get a brief
overview of \MiniSearch, then head straight to Sec.~\ref{sec:quickStart}
that gives a quick introduction with examples. 

If you are looking for more detailed information on the \MiniSearch
language and would like to write your own meta-search approaches with
it, then have a look at Sec.~\ref{sec:minisearch} that describes the 
components of \MiniSearch in detail, and check out Sec.~\ref{sec:examples}
that discusses various \MiniSearch examples with increasing difficulty and 
Sec.~\ref{sec:ownsearch} that provides tipps for writing your 
own meta-search in \MiniSearch.

\section{What is MiniSearch?}
\label{sec:intro}
\MiniSearch is a language for specifying {\em meta-search}. Meta-search is any
search approach that aims to find a good (or the best) solution,
such as Branch-and-Bound search (BaB) or Large Neighbourhoud 
Search (LNS). This means that with MiniSearch you can solve your 
\MiniZinc model with different kinds of meta-search, such as LNS.

The \MiniSearch language is based on \MiniZinc, and a \MiniSearch
specification is part of the \MiniZinc model. \MiniSearch also provides
a library of predefined meta-searches that can be directly used.

\section{Quick Start}
\label{sec:quickStart}
\subsection{Using \MiniSearch in your \MiniZinc model}
How to use \MiniSearch is best demonstrated on an example. Consider the \MiniZinc model below of the Golomb Ruler Problem.
The Golomb Ruler Problem is concerned with finding a ruler of minimal length with $m$ marks,
where the distances between all marks are different. 
\begin{lstlisting}
% Standard Golomb Ruler model
include "globals.mzn";

int: m;          % number of marks on the ruler
int: n = m*m;    % upper bound for length of ruler

array[1..m] of var 0..n: mark;  % the position of each mark
array[int] of var 0..n: differences =
    [ mark[j] - mark[i] | i in 1..m, j in i+1..m];

constraint mark[1] = 0;
constraint forall ( i in 1..m-1 ) ( mark[i] < mark[i+1] );
constraint alldifferent(differences);
constraint differences[1] < differences[(m*(m-1)) div 2]; % Sym breaking

solve :: int_search(mark, input_order, indomain, complete)%*\label{ex1:solve}*)
    minimize mark[m];

output ["golomb = ", show(mark), "\n"];
\end{lstlisting}
The \mzninline|solve| statement in line~\ref{ex1:solve} states the
objective (minimizing the position of the last mark) and provides 
a CP search strategy annotation\footnote{Note that the CP search strategy is a {\em tree search} strategy and not a meta-search strategy} (\mzninline|int_search|).
This will evoke the solver to solve this problem using the CP search strategy 
(if available), and using its standard approach for minimization (e.g. branch-and-bound for CP solvers).

Now let's extend this model with MiniSearch to solve it with LNS.
%This model can be easily extended with MiniSearch and thereby solved with
%different solving approaches. 
The simplest way is to use the built-in LNS definition that MiniSearch supplies, as in the example below:
\begin{lstlisting}
% Golomb Ruler model with Large Neighbourhood Search (LNS)
include "globals.mzn";
include "minisearch.mzn";%*\label{ex2:library}*)

int: m;          % number of marks on the ruler
int: n = m*m;    % upper bound for length of ruler

array[1..m] of var 0..n: mark;   % the position of each mark
array[int] of var 0..n: differences =
    [ mark[j] - mark[i] | i in 1..m, j in i+1..m];

constraint mark[1] = 0;
constraint forall ( i in 1..m-1 ) ( mark[i] < mark[i+1] );
constraint alldifferent(differences);   
constraint differences[1] < differences[(m*(m-1)) div 2]; % Sym breaking

int: lns_iterations;      %*\label{ex2:constStart}*)
float: destruction_rate;  
int: timeout_ms;          %*\label{ex2:constEnd}*)

solve :: int_search(mark, input_order, indomain, complete)%*\label{ex2:solve}*)
    search lns_min(mark[m], mark, lns_iterations, %*\label{ex2:lns}*)
                   destruction_rate, timeout_ms);

output ["golomb = ", show(mark), "\n"];
\end{lstlisting}

There are two changes in this model.
First, the library \mzninline|minisearch.mzn| is included (line~\ref{ex2:library}). This
is always necessary when using MiniSearch, since it defines all necessary built-ins. 
%This library contains definitions of meta-searches, such as LNS, that can simply be included in the model.
Second, in line~\ref{ex2:lns}, we use the \mzninline|search| keyword
followed by the call to the MiniSearch built-in \mzninline|lns_min|. We declare its constant arguments 
in lines\ref{ex2:constStart}-~\ref{ex2:constEnd}.
The built-in  \mzninline|lns_min| has  the following signature:
\begin{lstlisting}
function ann: lns_min (var int: obj, 
                       array[int] of var int: x,
                       int: iterations, 
                       float: d,
                       int: timeout_ms);
\end{lstlisting}
where \mzninline|obj| is the objective variable that is sought to be minimized,
\mzninline|x| is the array of search variables on which LNS will be performed,
\mzninline|iterations| is the number of LNS iterations, \mzninline|d| is the 
destruction rate, i.e. the percentage of variables in \mzninline|x| that 
should be destroyed (freed) in each LNS iteration, and finally, 
\mzninline|timeout_ms| is the timeout in milliseconds for each 
LNS iteration. Below is a sample data file for the LNS Golomb Ruler model.
\begin{lstlisting}
% Sample data file for Golomb Ruler with LNS
m = 19;                    
lns_iterations = 100;       % 100 LNS iterations
destruction_rate = 0.2;     % 20% destruction rate per iteration
timeout_ms = 6*1000;        % 6 seconds timeout per iteration
\end{lstlisting}

\subsection{Running MiniSearch in your MiniZinc model}


The simplest way to run your MiniZinc model with MiniSearch
is by using a solver that can take FlatZinc as input, such 
as Gecode, Choco or or-tools. These solvers provide executables
such as \mzninline|fzn-gecode| or \mzninline|fzn_choco|.


\subsubsection{Running MiniSearch with a FlatZinc solver}

If you are using the command line, use the \mzninline|mzn-fzn-lite|
executable and execute it in the following way:
\begin{lstlisting}[language=bash]
mzn-fzn-lite --solver <fzn-solver> <my-model.mzn> <my-data.dzn>
\end{lstlisting}
where \mzninline|<fzn-solver>| stands for the FlatZinc
solver (don't forget to add the path if it is not in the PATH), 
\mzninline|<my-model.mzn>| stands for the MiniZinc model that
contains a MiniSearch specification, and \mzninline|<my-data.dzn>|
is the optional data file. As an example, if we write the 
Golomb Ruler model with LNS into the file \mzninline|golomb-lns.mzn|
and write the sample data specification from above into 
\mzninline|golomb.dzn|, and we want to run it on \mzninline|fzn_choco|,
we execute the following command:
\begin{lstlisting}[language=bash]
mzn-fzn-lite --solver fzn_choco golomb-lns.mzn golomb.dzn
\end{lstlisting}

\subsubsection{Running MiniSearch with an incremental MiniSearch solver}

Solvers can also directly support MiniSearch, by implementing its API.
So far, Gecode supports this option and it is shipped with the distribution
of MiniSearch as the executable \mzngecodelite.

In the command line, it can be executed in the following way:
\begin{lstlisting}[language=bash]
%*\mzngecodelite    <my-model.mzn> <my-data.dzn>
\end{lstlisting}


\section{The MiniSearch Language}
\label{sec:minisearch}

\MiniSearch is a high level meta-search language based on \MiniZinc 2.0. In \MiniSearch, \MiniZinc constructs are used for formulating the model and the constraints posted during search, and special \MiniSearch built-ins are used for specifying the search.

A \MiniSearch specification is provided, together with the \mzninline|search| keyword, as an argument to \MiniZinc's \mzninline|solve| item. It is executed by the \MiniSearch kernel.
The \MiniSearch built-ins are summarised in Tab.~\ref{tab:builtins}, but we discuss them in more detail
in the following sections.

Using these built-ins, users can define meta-search functions such as the following branch-and-bound (BaB) minimisation:
\begin{lstlisting}
include "minisearch.mzn"; %*\label{ex:incl}*)
var int: obj; % other variables and constraints not shown
solve search minimize_bab(obj); %*\label{ex:solve}*)
output ["Objective: "++show(obj)];  %*\label{ex:output}*)

function ann: minimize_bab(var int: obj) =
  repeat( if next() then commit() /\ print() /\ post(obj < sol(obj)) %*\label{ex:specb}*)
          else break endif ); %*\label{ex:spec}*)
\end{lstlisting}
The \mzninline|include| item on line~\ref{ex:incl} includes the built-in
\MiniSearch function declarations. This is necessary for any \MiniZinc model that uses \MiniSearch.
Line~\ref{ex:solve} contains the \MiniZinc \mzninline|solve|
item followed by the new \mzninline|search| keyword and a user-defined \MiniSearch function that
takes a variable representing the objective as argument. Line~\ref{ex:output} is the
\MiniZinc \mzninline|output| item, specifying how solutions should be
printed. Lines~\ref{ex:specb}--\ref{ex:spec} contain the actual \MiniSearch specification. We will explain the different built-ins in more detail below, but the specification can be read as follows: repeatedly try to find the next solution; and if that is successful, 
commit to the solution, print it and add the constraint that the objective must have a lower value than the current solution.
If unsuccessful, break out of the repeat.

\begin{table}[b!]
\centering
\footnotesize
\begin{tabular}{ll}
\hline
\MiniSearch built-ins & Description \\
\hline
\mzninline|next()| & find the next solution \\ 
\mzninline|post(|$c$\mzninline|)| & post the \MiniZinc constraint $c$ in the current scope\\
\mzninline|scope(|$s$\mzninline|)| & open a local scope containing search $s$ \\  
\hline
$s_1$ \mzninline|/\| $s_2$ & run $s_1$ and iff successful, run $s_2$ \\ 
$s_1$ \mzninline|\/| $s_2$ & run $s_1$ and iff it fails, run $s_2$ \\ 
\mzninline|if |$s$\mzninline| then |$s_1$\mzninline| else | $s_2$ & if $s$ is successful, run $s_1$, otherwise $s_2$\\  
\mzninline|repeat(|$s$\mzninline|)| & repeat search $s$ until \mzninline|break| is executed \\  
\mzninline|repeat (i in 1..|$N$\mzninline|)(|$s$\mzninline|)| & repeat search $s$ $N$ times or until \mzninline|break| is executed \\  
\mzninline|break()| & break within a \mzninline|repeat| \\ 
\mzninline|fail()| & return `failure' \\  
\mzninline|skip()| & return `success' \\  
\mzninline|time_limit(|$ms$,$s$\mzninline|)| & run $s$ until timelimit $ms$ is reached \\ 
\hline
\mzninline|print(|$S$\mzninline|)| & print \MiniZinc output string $S$\\
\mzninline|print()| & print solution according to model output specification\\
$c$ \mzninline|:=| $v$& assign parameter $c$ the value $v$\\
\mzninline|commit()| & commit to last found solution in function scope\\  
\mzninline|sol(|$v$\mzninline|)| & return solution value of variable $v$\\
\mzninline|hasSol()| & returns true if a solution has been found\\
\hline
\end{tabular}
\label{tab:builtins}
\caption{\MiniSearch built-ins}
\end{table}



\subsection{\MiniSearch built-ins involving the solver}
Communication with the solver is restricted to three forms: invoking the solver, adding constraints/variables to the model, and scopes for temporary variables and constraints.

\subsubsection{Invoking the solver} 
The \MiniSearch instruction for finding the next solution is \mzninline|next()|. 
It is successful if a solution has been found, and fails otherwise. 
The variable/value labelling strategy (such as first-fail on the smallest domain value) can be set in two ways: either
by annotating the \mzninline|solve| item (as in standard \MiniZinc), which sets the labelling globally, for every call to \mzninline|next()|.
Otherwise, by annotating any \MiniSearch function call, such as \mzninline|minimize_bab|, with a labelling strategy. Note, however,
that as in \MiniZinc, solvers may ignore these annotations, for example if the labelling is not supported.

Solvers may declare support for native meta-search strategies, as with global constraints, in which case these \MiniSearch functions are treated as built-ins.


\subsubsection{Adding constraints and variables} 
A constraint is added by calling the \mzninline|post()| built-in with a constraint as argument. Constraints can be formulated using the same \MiniZinc constructs as in the model, including global constraints, user-defined functions and predicates.
Variables can be dynamically added during search too, using the \MiniZinc \mzninline|let| construct).

\subsubsection{Search Scopes} 
Search scopes define the lifespan of constraints and variables in the model. \MiniSearch has an implicit {\em global} search scope that contains all variables and constraints of the model. A new search scope can be created by using the \mzninline|scope(s)| built-in that takes a \MiniSearch specification \mzninline|s| as an argument. When entering a scope, search is assumed to start from the root again. Whenever the execution leaves a scope, all constraints and variables that were added in the scope are removed from the model and the solver. Execution in the enclosing scope resumes from the point where it left off.



\subsection{\MiniSearch control built-ins}\label{sec:control}

All \MiniSearch built-ins have an implicit return value that represents either `success' ({\em true}) or `failure' ({\em false}).
Using this concept, we introduce \MiniSearch control built-ins.
All built-ins execute their arguments in order.

\subsubsection{And, Or, Repeat}
The \mzninline|/\|-built-in runs its arguments in order 
and stops to return \textit{false} as soon as one of its arguments fails. 
Similarly, the \mzninline|\/|-built-in stops and returns \textit{success} as soon as one of its arguments succeeds.
Existing control mechanisms of \MiniZinc such as \mzninline|if then else endif| expressions can be used as well. 
The \mzninline|repeat(s)| built-in takes a \MiniSearch specification
\mzninline|s| and repeats it until a \mzninline|break| built-in is
executed; returns \emph{false} if a break happened, otherwise returns what $s$ returned.
 The delimited variant \mzninline|repeat(i in 1..N)(s)| will execute \mzninline|s| for \mzninline|N| iterations (or until \mzninline|break| is executed).


\subsubsection{Time-Limits}
The built-in \mzninline|time_limit(ms,s)| imposes a time limit \mzninline|ms| (in milliseconds) on any \MiniSearch specification \mzninline|s|. This way, \mzninline|s| stops whenever the time limit is reached, returning its current status.
Time-limits are handled transparently by the \MiniSearch kernel as an exception.

\subsubsection{Assigning values to constants}
In standard \MiniZinc constant parameters such as \mzninline|int: N=10;| cannot change their value. However, in \MiniSearch we often want to change constants across different iterations. For this purpose, we added the assignment operator \mzninline|:=| which may only be used inside a \MiniSearch specification. It overwrites that constant's current value by the value supplied.



\subsection{Solution management}

The strength of any meta-search language lies in using intermediate solutions to guide the remaining search. For instance, branch-and-bound needs to access the objective to post further constraints, and a Large Neighbourhood Search thaws some of the variables in a solution to continue in that neighbourhood.

To facilitate working with solutions, the most recently found solution is always accessible in \MiniSearch using the \mzninline|sol| built-in, where \mzninline|sol(x)| returns the value of \mzninline|x| in the last solution. \MiniSearch also provides a \mzninline|hasSol()| built-in to test whether a solution exists.

\subsubsection{User-defined functions}
When a \MiniSearch strategy is defined as a \MiniZinc function, a local solution scope is created. This means that any solution found by a call to \mzninline|next()| inside the function is visible for the code in the function body, but not for the caller of the function when the function returns. 
This architecture allows for calls to \mzninline|next()| to be \emph{encapsulated}, i.e., a function can make ``tentative'' calls to next in a nested search scope and only commit if these succeed. Sect.~\ref{sec:examples} shows how AND/OR search can be implemented based on this principle.
In order to make the current solution accessible to the caller, the function must call the \mzninline|commit| built-in. A function returns `success' if it called \mzninline|commit()| at least once, and `failure' otherwise, and the last solution committed by the function will then become the current solution of the caller. 

\subsubsection{Printing Solutions \& Debugging}
The \mzninline|print()| function without any arguments prints the last found solution in the format specified in the model's \mzninline|output| item. Alternatively, \mzninline|print(s)| provides more fine-grained control over the output. It prints the string \mzninline|s|, which can be constructed dynamically from values in the solution using calls to \mzninline|sol|.
\MiniSearch can be debugged using \mzninline|print()| and \MiniZinc's \mzninline|trace()| function to display values of parameters, variables, and 
arbitrary expressions during search. Furthermore, the \MiniSearch interpreter uses the C++ stack, so C++ debuggers can be used to follow the meta-search.




\section{\MiniSearch Examples}
\label{sec:examples}

In the following we present different meta-search examples, increasing in
difficulty to illustrate the different \MiniSearch built-ins.

\subsection{Printing all solutions}
The example below shows the \MiniSearch function that finds all solutions and prints them.
\begin{lstlisting}
function ann: print_all_solutions() =
  repeat (
    if next() then print()
    else break endif
  );
\end{lstlisting}
The body of \mzninline|repeat| is executed until \mzninline|break|
is reached: whenever the solver found another solution via \mzninline|next()|,
it is printed with \mzninline|print()|. If no solution can be found, 
\mzninline|repeat| stops. Note that we do not commit to the found solution
with \mzninline|commit| and therefore the last found solution will not be
accessible outside of the function's scope.
This \MiniSearch function can be called in a \MiniZinc model by replacing
the standard \mzninline|solve satisfy| or \mzninline|solve minimize| item with:
\begin{lstlisting}
solve search print_all_solutions();
\end{lstlisting}


\subsection{Printing first $n$ solutions}
Let's extend the previous example to a function that prints all
first $n$ solutions:
\begin{lstlisting}
function ann: print_first_n_solutions(int: n) =
  repeat (i in 1..n) (
    if next() then print()
    else break endif
  );
\end{lstlisting}
The \mzninline|repeat| builtin will stop either after 
$n$ iterations have been reached, or \mzninline|break| has been executed
(which happens in case there are less than $n$ solutions).
We can call this function similarily with:
\begin{lstlisting}
solve search print_first_n_solutions(10);
\end{lstlisting}
which will print the first 10 solutions.


\subsection{Restart Branch-and-Bound}

\begin{lstlisting}
function ann: restart_bab_min(var int: obj) =
  repeat (
    scope(
      if hasSol() then post(obj < sol(obj)) 
      else skip endif
      /\ 
      if next() then commit()
      else break endif
    )
  );
\end{lstlisting}


\section{How to write your own meta-search in MiniSearch}
\label{sec:ownsearch}

\end{document}
