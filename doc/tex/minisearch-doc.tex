\documentclass[a4paper,13pt,onecolumn]{article}%{memoir}

\usepackage{listings}
\usepackage{hyperref}

\ifx\pdftexversion\undefined
 % \usepackage[dvips]{graphicx}
  \usepackage[dvips,usenames]{color}
\else
 % \usepackage[pdftex]{graphicx}
  \usepackage[pdftex,usenames,dvipsnames]{color}
\fi

% MiniSearch syntax for lstlistings

\definecolor{lightgray}{rgb}{0.97, 0.97, 0.97}
%\definecolor{lightgray}{rgb}{0.83, 0.83, 0.83}
%\definecolor{orange}{HTML}{FF7F00}

% Syntax highlighting for stochastic Minizinc in listings
\lstdefinelanguage{minizinc}
{
morekeywords={
% minizinc keywords
ann, annotation, any, array, assert, bool, constraint, else, endif, enum, float, forall, function,
if, in, include, int, list, of, op, output, par, predicate, record, set,
solve, string, test, then, tuple, type, var, where,
%minimize, maximize % removed for minisearch
% minizinc functions
abort, abs, acosh, array_intersect, array_union,
array1d, array2d, array3d, array4d, array5d, array6d, asin, assert, atan, bool2int, card,
ceil, combinator, concat, cos, cosh, dom, dom_array, dom_size, dominance,
fix, exp, floor, index_set, index_set_1of2,
index_set_2of2, index_set_1of3, index_set_2of3, index_set_3of3, int2float, is_fixed,
join, lb, lb_array, length, let, ln, log, log2, log10, min, max, pow, product, round, set2array,
show, show_int, show_float, sin, sinh, sqrt, sum, tan, tanh, trace, ub, and ub_array,
% minisearch keywords
minisearch, search, repeat, next, commit, print, post, sol, scope, time_limit, break, fail, skip
},
sensitive=false, % are the keywords case sensitive
%morecomment=[l][\bfseries\color{OrangeRed}]{::},
morecomment=[l][\em\color{ForestGreen}]{\%},
%morecomment=[s]{/*}{*/},
morestring=[b]",
}

% settings for listings
\lstset{ %
  backgroundcolor=\color{lightgray},   % choose the background color; you must add \usepackage{color} or \usepackage{xcolor}
  basicstyle=\scriptsize\ttfamily,        % the size of the fonts that are used for the code
  belowskip=-2em,
  breakatwhitespace=false,         % sets if automatic breaks should only happen at whitespace
  breaklines=true,                 % sets automatic line breaking
  captionpos=b,                    % sets the caption-position to bottom
  commentstyle=\color{ForestGreen},    % comment style
%  deletekeywords={...},            % if you want to delete keywords from the given language
  escapeinside={\%*}{*)},          % if you want to add LaTeX within your code
  extendedchars=true,              % lets you use non-ASCII characters; for 8-bits encodings only, does not work with UTF-8
  frame=single,                    % adds a frame around the code
  keepspaces=true,                 % keeps spaces in text, useful for keeping indentation of code (possibly needs columns=flexible)
  keywordstyle=\bfseries\color{blue},       % keyword style
  language=minizinc,                % the language of the code
%  morekeywords={*,...},            % if you want to add more keywords to the set
  numbers=left,                    % where to put the line-numbers; possible values are (none, left, right)
  numbersep=5pt,                   % how far the line-numbers are from the code
  numberstyle=\tiny\color{Gray}, % the style that is used for the line-numbers
  rulecolor=\color{black},         % if not set, the frame-color may be changed on line-breaks within not-black text (e.g. comments (green here))
  showspaces=false,                % show spaces everywhere adding particular underscores; it overrides 'showstringspaces'
  showstringspaces=false,          % underline spaces within strings only
  showtabs=false,                  % show tabs within strings adding particular underscores
  stepnumber=1,                    % the step between two line-numbers. If it's 1, each line will be numbered
  stringstyle=\color{Red},     % string literal style
  tabsize=2,                       % sets default tabsize to 2 spaces
  title=\lstname                   % show the filename of files included with \lstinputlisting; also try caption instead of title
}

%\def\mzninline{\lstinline[basicstyle=\ttfamily,annotationstyle=\normalfont]}
\def\mzninline{\verb}


% setting font family (uncomment if you want default font)
\renewcommand*\rmdefault{ppl}
%\renewcommand*{\familydefault}{\sfdefault}

\begin{document}
\title{MiniSearch Manual}
\author{Andrea Rendl \and Guido Tack}
%National ICT Australia (NICTA) and Faculty of IT, Monash University,
%Australia

\maketitle

\section{Overview}
This is the documentation and manual for using MiniSearch. If you 
are already familiar with using MiniZinc and want to get a brief
overview of MiniSearch, then head straight to Sec.~\ref{sec:quickStart}
that gives a quick introduction with examples.

If you are looking for more detailed information on the MiniSearch
language and would like to write your own meta-search approaches with
it, then have a look at Sec.~\ref{sec:minisearch} that describes the 
components of MiniSearch in detail, and check out Sec.~\ref{sec:ownsearch}
that provides a small tutorial on how to write your own meta-search
on a couple of examples.


\section{Quick Start}
\label{sec:quickStart}
MiniSearch is a language for specifying {\em meta-search}. Meta-search is any
search approach that aims to find a good (or the best) solution,
such as Branch-and-Bound search (BaB) or Large Neighbourhoud 
Search (LNS). This means that with MiniSearch you can solve your 
MiniZinc model with different kinds of meta-search, such as LNS.

This is best demonstrated on an example. Consider the MiniZinc model below of the Golomb Ruler Problem.
The Golomb Ruler Problem is concerned with finding a ruler of minimal length with $m$ marks,
where the distances between all marks are different. 
\begin{lstlisting}
% Standard Golomb Ruler model
include "globals.mzn";

int: m;          % number of marks on the ruler
int: n = m*m;    % upper bound for length of ruler

array[1..m] of var 0..n: mark;  % the position of each mark
array[int] of var 0..n: differences =
    [ mark[j] - mark[i] | i in 1..m, j in i+1..m];

constraint mark[1] = 0;
constraint forall ( i in 1..m-1 ) ( mark[i] < mark[i+1] );
constraint alldifferent(differences);
    % Symmetry breaking
constraint differences[1] < differences[(m*(m-1)) div 2];

solve :: int_search(mark, input_order, indomain, complete)%*\label{ex1:solve}*)
    minimize mark[m];

output ["golomb = ", show(mark), "\n"];
\end{lstlisting}
The \mzninline|solve| statement in line~\ref{ex1:solve} states the
objective (minimizing the position of the last mark) and provides 
a CP search strategy annotation\footnote{Note that the CP search strategy is a {\em tree search} strategy and not a meta-search strategy} (\mzninline|int_search|).
This will evoke the solver to solve this problem using the CP search strategy 
(if available), and using its standard approach for minimization (e.g. branch-and-bound for CP solvers).

Now let's extend this model with MiniSearch to solve it with LNS.
%This model can be easily extended with MiniSearch and thereby solved with
%different solving approaches. 
The simplest way is to use the built-in LNS definition that MiniSearch supplies, as in the example below:
\begin{lstlisting}
% Golomb Ruler model with Large Neighbourhood Search (LNS)
include "globals.mzn";
include "minisearch.mzn";%*\label{ex2:library}*)

int: m;          % number of marks on the ruler
int: n = m*m;    % upper bound for length of ruler

array[1..m] of var 0..n: mark;   % the position of each mark
array[int] of var 0..n: differences =
    [ mark[j] - mark[i] | i in 1..m, j in i+1..m];

constraint mark[1] = 0;
constraint forall ( i in 1..m-1 ) ( mark[i] < mark[i+1] );
constraint alldifferent(differences);
    % Symmetry breaking
constraint differences[1] < differences[(m*(m-1)) div 2];

int: lns_iterations;      %*\label{ex2:constStart}*)
float: destruction_rate;  
int: timeout_ms;          %*\label{ex2:constEnd}*)

solve :: int_search(mark, input_order, indomain, complete)%*\label{ex2:solve}*)
    search lns_min(mark[m], mark, lns_iterations, destruction_rate, timeout_ms) ;%*\label{ex2:lns}*)

output ["golomb = ", show(mark), "\n"];
\end{lstlisting}

There are two changes in this model.
First, the library \mzninline|minisearch.mzn| is included (line~\ref{ex2:library}). This
is always necessary when using MiniSearch, since it defines all necessary built-ins. 
%This library contains definitions of meta-searches, such as LNS, that can simply be included in the model.
Second, in line~\ref{ex2:lns}, we use the \mzninline|search| keyword
followed by the call to the MiniSearch built-in \mzninline|lns_min|. We declare its constant arguments 
in lines\ref{ex2:constStart}-~\ref{ex2:constEnd}.
The built-in  \mzninline|lns_min| has  the following signature:
\begin{lstlisting}
function ann: lns_min (var int: obj, 
                       array[int] of var int: x,
                       int: iterations, 
                       float: d,
                       int: timeout_ms);
\end{lstlisting}
where \mzninline|obj| is the objective variable that is sought to be minimized,
\mzninline|x| is the array of search variables on which LNS will be performed,
\mzninline|iterations| is the number of LNS iterations, \mzninline|d| is the 
destruction rate, i.e. the percentage of variables in \mzninline|x| that 
should be destroyed (freed) in each LNS iteration, and finally, 
\mzninline|timeout_ms| is the timeout in milliseconds for each 
LNS iteration. Below is a sample data file for the LNS Golomb Ruler model.
\begin{lstlisting}
% Sample data file for Golomb Ruler with LNS
m = 19;                    
lns_iterations = 100;       % 100 LNS iterations
destruction_rate = 0.2;     % 20% destruction rate per iteration
timeout_ms = 6*1000;        % 6 seconds timeout per iteration
\end{lstlisting}


\section{The MiniSearch Language}
\label{sec:minisearch}


\section{How to write your own meta-search in MiniSearch}
\label{sec:ownsearch}

\end{document}
